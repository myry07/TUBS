\documentclass[12pt, letterpaper]{article}
\usepackage[utf8]{inputenc} % 用于输入编码设置
\usepackage{CJKutf8} %中文
\usepackage{graphicx} %图片包
\usepackage{xcolor} %字体颜色
\usepackage{caption} % 加载 caption 包 去掉Figure 1
\usepackage{amsmath} %数学包
\usepackage{amsmath}



\begin{document}
\begin{CJK*}{UTF8}{gbsn}%显示中文% gbsn 宋体
% gkai 楷体

\title{Zusammenfassung des OR}
\author{Jiaqi Wang} % 可以在此处添加作者名
\date{07.01.2024} % 此命令插入当前日期
\maketitle % 这个命令会插入标题、作者和日期
\pagenumbering{gobble} %隐藏页码

\vspace{5cm}

\section{Einführung} % 文章第一节,引言

\begin{itemize}

\item OR: Entwicklung und der Einsatz von mathematischen Verfahren zur Unterstützung von (betriebswirtschaftlichen) Entscheidungsprozessen

\item Mathematisches Modell \& Problemdefinition:
\begin{enumerate}
\item Zielfunktion: Was wollen wir erreichen?
\item Nebenbedingungen: Worauf müssen wir achten?
\item Entscheidungsvariablen: Was können wir ändern?
\end{enumerate}

\end{itemize}

\newpage
%%%%%%%%%%%%%
\section{Simplex-Algorithmus}
Frage: 如何解决线性问题?
\subsection{Simplex}

\begin{itemize}
\item Definitionen

\begin{enumerate}
\item Strukturvariablen: 矩阵第一行的前几个 $x_1, x_2...$
\item Schlupfvariablen: 矩阵第一行的后几个 $x_4, x_5...$

\item Basis: 矩阵变换后 不等于0的
\item NB: 矩阵变换后 等于0的\\
\textbf{\textcolor{red}{Schlupf与Struktur不会变化 B与NB会变化}}

\item Restriktionsvektor: 矩阵最后一列 除了\textbf{\textcolor{blue}{ZFW}}

\end{enumerate}


\item primaler Simplex计算过程
\begin{enumerate}
\item 找\textbf{\textcolor{red}{最大}}的ZF系数 (矩阵第一行) \\
该ZF系数所在的列 -$>$ Pivotsplate

\item 该列所有的系数 (不要ZF系数 \& $\geq0$)被RS除\\
结果\textbf{\textcolor{red}{最小}}的-$>$ Pivotelemet

\item 行列式化简 Pivotsplate 只有 Pivotelemet = 1 其余都=0

\item 循环 直到ZF系数 $\geq0$


\end{enumerate}

\end{itemize}



\newpage
%%%%%%%%%%%%%%
\section{Sensitivitätsanalyse}
Frage: RBn轻微改变会发生什么? u.d.N. 中的 Engpässe

\subsection{Ökonomische Interpretation}
\begin{itemize}
\item 无Engpass: optimal时 Schlupf$\neq$0 则该NB无Engpass
\item 有Engpass: optimal时 Schlupf=0 则该NB有Engpass\\
\underline{Schlupf=0代表该NB到达极限了 不能更进一步了 所以瓶颈}

\item Schattenpreis: 某个NB的松弛 对于ZF的变化率\\
Endtableau中 Schlupf的系数

\end{itemize}

\subsection{Sensitivitätsanalyse}
观察ZF系数 与 RS系数变化的反应
\begin{center}
\textcolor{red}{RS -$>$ Schlupf} \ \textcolor{blue}{ZF -$>$ Struktur}
\end{center}

\begin{itemize}
\item 是BV 在行中找
\begin{enumerate}
\item $\underline{\lambda_k}$ = RS系数 (因为出现在End矩阵某行中)
\item $\overline{\lambda_k} = \infty$ 

\item $\underline{\mu_k}$ = \textcolor{red}{-}ZF/行 (行中出现了)
\item$\overline{\mu_k} = \infty$


\end{enumerate}

\item 是NBV 在列中找
\begin{enumerate}
\item $\underline{\lambda_k}$ = RS/列 (与Pivotelemet算法一致 $>0$) \\
若无非负列 $\underline{\lambda_k} = \infty$ 
\item $\overline{\lambda_k}$ = \textcolor{red}{-}RS/列 (与Pivotelemet算法一致 $<0$)\\若无负列 $\underline{\lambda_k} = \infty$ 

\item $\underline{\mu_k} = \infty$ 
\item  $\overline{\mu_k} = $ ZF系数 (因为行中没有 只有列才有)

\end{enumerate}
\end{itemize}


\newpage
%%%%%%%%%%%%
\section{Dualer Simplex \& Dualität}

\begin{align*}
\text{Frage:} &\quad \text{一般情况下,Starttableau 中 ZF 的 Schlupf} = 0 \\
&\quad \text{如果 } 0 \text{ 不是可行解会怎样?}
\end{align*}

\subsection{Duales Programm}
\begin{itemize}
\item Definition
\begin{enumerate}
\item primales Modell: 原始模型 生产者视角 ZF \textcolor{red}{max} Deckungsbeitrag
\item duales Modell: 对偶模型 购买者视角 ZF \textcolor{red}{min} Gesamtkaufpreis
\end{enumerate}


\item 二者矩阵关系
\begin{enumerate}
\item primal的Struktur = \textcolor{red}{-} dual的RS转秩
\item primal的RS = \textcolor{red}{-} dual的Struktur转秩
\item Struktur \& NBV 的矩阵 互为转秩 并 $(\cdot -)$

\end{enumerate}

\item 何时用Dual
\begin{enumerate}
\item RS$<$0 \& ZF系数小于0
\\很好理解 根据二者矩阵关系 一转换可知 priaml时 RS ZF都$>$0

\end{enumerate}

\item dualer Simplex计算过程
\begin{enumerate}
\item 找到最小的非正RS -$>$ Pivotzeile
\item ZF/该行的x (该x$<0$) 最小的是 Pivotelemet
\item 之后与primaler一致 \ 直至RS与ZF $>0$
\item 若RS$>$0 \ 但ZF$<$0 用primaler Simplex
\end{enumerate}
\end{itemize}


\newpage
%%%%%%%%%%%
\section{Ganzzahlige Programmierung \& Branch and Bound}
Frage: 比如寻找人数必须是整数

\begin{itemize}
\item Branch and Bound Verfahren

\begin{enumerate}
\item 用Simplex计算出 交点与值($x_1$,$x_2$,ZFW)
\item 选择其中的非整数 向下取整\&向下取整+1 并且求出各自的ZFW
\item 直到ZFW是整数\&最大 

\end{enumerate}
\end{itemize}


\newpage
%%%%%%%%%%%%
\section{Mehrfache Zielsetzung \& Modellierungstechniken}
Frage: 有多个Ziel如何解决?

\begin{itemize}
\item Mehrzieloptimierung

\begin{enumerate}
\item 对Ziel排序
\item 对于Ziel A 直接用Simplex计算出 $x_1, x_2...ZFW$
\item 对于Ziel B 将Ziel的结果代入NB\ 再用Simplex
\item 以此类推



\end{enumerate}
\end{itemize}



\end{CJK*}
\end{document}